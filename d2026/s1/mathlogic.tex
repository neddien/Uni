% Created 2025-12-18 Thu 19:05
% Intended LaTeX compiler: pdflatex
\documentclass[11pt]{article}
\usepackage[utf8]{inputenc}
\usepackage[T1]{fontenc}
\usepackage{graphicx}
\usepackage{longtable}
\usepackage{wrapfig}
\usepackage{rotating}
\usepackage[normalem]{ulem}
\usepackage{amsmath}
\usepackage{amssymb}
\usepackage{capt-of}
\usepackage{hyperref}
\usepackage{mathptmx}  % Times font
\usepackage{helvet}   % Helvetica font
\renewcommand{\familydefault}{\sfdefault} % Sans-serif as default
\usepackage{titlesec}
\usepackage{lmodern}
\usepackage{tikz}
\usepackage{pgfplots}
\usepackage{karnaugh-map}
\pgfplotsset{compat=1.18}
\author{Serob Tigranyan}
\date{\today}
\title{Mathematical Logic}
\hypersetup{
 pdfauthor={Serob Tigranyan},
 pdftitle={Mathematical Logic},
 pdfkeywords={},
 pdfsubject={},
 pdfcreator={Emacs 30.2 (Org mode 9.7.39)}, 
 pdflang={English}}
\begin{document}

\maketitle
\tableofcontents

\newpage
\section{Proving Tautology using Identical Transformations}
\label{sec:orga8b28cc}
Using identical transformations prove that the following formula is a tautology:
\[
P \to (Q \to ((P \lor Q) \to (P \land Q)))
\]
\subsection{Understanding Tautology}
\label{sec:org4f24f18}
A formula is a tautology if it evaluates to true for all possible truth value assignments. We'll use logical equivalences to transform the formula.
\subsection{Applying Transformations}
\label{sec:org93d7b9f}
We begin with the implication equivalence: \(A \to B \equiv \neg A \lor B\)

Starting with the formula:
\[
P \to (Q \to ((P \lor Q) \to (P \land Q)))
\]

Apply implication elimination to the outermost implication:
\[
\equiv \neg P \lor (Q \to ((P \lor Q) \to (P \land Q)))
\]

Apply implication elimination to the next implication:
\[
\equiv \neg P \lor (\neg Q \lor ((P \lor Q) \to (P \land Q)))
\]

Apply implication elimination to the innermost implication:
\[
\equiv \neg P \lor (\neg Q \lor (\neg(P \lor Q) \lor (P \land Q)))
\]

By De Morgan's law: \(\neg(P \lor Q) \equiv \neg P \land \neg Q\)
\[
\equiv \neg P \lor (\neg Q \lor ((\neg P \land \neg Q) \lor (P \land Q)))
\]

\newpage
Applying associativity of disjunction:
\[
\equiv \neg P \lor \neg Q \lor (\neg P \land \neg Q) \lor (P \land Q)
\]

By absorption law: \(A \lor (A \land B) \equiv A\), we observe that \(\neg P \lor (\neg P \land \neg Q) \equiv \neg P\):
\[
\equiv \neg P \lor \neg Q \lor (P \land Q)
\]

Now we can factor this expression. Notice that:
\[
\neg P \lor \neg Q \lor (P \land Q) \equiv (\neg P \lor \neg Q) \lor (P \land Q)
\]

By De Morgan's law in reverse: \(\neg P \lor \neg Q \equiv \neg(P \land Q)\)
\[
\equiv \neg(P \land Q) \lor (P \land Q)
\]

This is the law of excluded middle: \(\neg A \lor A \equiv \top\) (always true)
\[
\equiv \top
\]

Therefore, the formula is a \textbf{\textbf{tautology}}.
\[
\boxed{P \to (Q \to ((P \lor Q) \to (P \land Q))) \equiv \top}
\]

\newpage
\section{Finding the ANF (Zhegalkin Polynomial)}
\label{sec:orge98c39c}
Find the ANF (Zhegalkin polynomial) of the following Boolean function and determine whether it can be expressed as a linear Zhegalkin polynomial:
\[
f(x,y,z) = xyz \lor xy'z \lor x'yz' \lor x'y'z'
\]
\subsection{Constructing the Truth Table}
\label{sec:orgca50b09}
First, we construct the truth table for the given function:

\begin{center}
\begin{tabular}{|c|c|c|c|}
\hline
$x$ & $y$ & $z$ & $f(x,y,z)$ \\
\hline
0 & 0 & 0 & 1 \\
0 & 0 & 1 & 0 \\
0 & 1 & 0 & 1 \\
0 & 1 & 1 & 0 \\
1 & 0 & 0 & 0 \\
1 & 0 & 1 & 1 \\
1 & 1 & 0 & 0 \\
1 & 1 & 1 & 1 \\
\hline
\end{tabular}
\end{center}
\subsection{Computing the ANF using Pascal's Triangle Method}
\label{sec:org0e9821b}
To find the ANF, we use the XOR-based method. We create an auxiliary table where each row is computed by XORing adjacent values:

\begin{center}
\begin{tabular}{|c|c|c|c|c|c|c|c|c|}
\hline
Row & 000 & 001 & 010 & 011 & 100 & 101 & 110 & 111 \\
\hline
0 & 1 & 0 & 1 & 0 & 0 & 1 & 0 & 1 \\
1 & 1 & 1 & 1 & 0 & 1 & 1 & 1 & \\
2 & 0 & 0 & 1 & 1 & 0 & 0 & & \\
3 & 0 & 1 & 0 & 1 & & & & \\
4 & 1 & 1 & 1 & & & & & \\
5 & 0 & 0 & & & & & & \\
6 & 0 & & & & & & & \\
\hline
\end{tabular}
\end{center}

\newpage
\subsection{Reading the ANF Coefficients}
\label{sec:org07f2884}
The ANF coefficients correspond to the first column of each row. Reading from top to bottom, we get the coefficients for terms: \(1, x, y, xy, z, xz, yz, xyz\)

From our table: coefficients are \(1, 1, 0, 0, 1, 0, 0, 0\)

Therefore, the ANF is:
\[
\boxed{f(x,y,z) = 1 \oplus x \oplus z}
\]
\subsection{Determining Linearity}
\label{sec:orgd9247eb}
A Zhegalkin polynomial is linear if it contains only terms of degree 0 (constant) and degree 1 (single variables), with no products of variables.

Our ANF contains only the constant term \(1\) and the linear terms \(x\) and \(z\). Therefore:
\[
\boxed{\text{The function IS a linear Zhegalkin polynomial}}
\]

\newpage
\section{Constructing Truth Table of Boolean Function}
\label{sec:orga2a9fd6}
Construct the truth table of the following boolean function:
\[
f(x, y, z) = x' \to (x \leftrightarrow (y \oplus (xz)))
\]
\subsection{Understanding the Operators}
\label{sec:org4c0591a}
\begin{itemize}
\item \(\oplus\) is XOR (exclusive or): \(A \oplus B\) is true when exactly one of \(A\) or \(B\) is true
\item \(\leftrightarrow\) is biconditional (XNOR): \(A \leftrightarrow B\) is true when \(A\) and \(B\) have the same truth value
\item \(\to\) is implication: \(A \to B\) is false only when \(A\) is true and \(B\) is false
\end{itemize}
\subsection{Building the Truth Table}
\label{sec:orge94658f}
We evaluate the function step by step for each combination:

\begin{center}
\begin{tabular}{|c|c|c|c|c|c|c|c|c|}
\hline
$x$ & $y$ & $z$ & $x'$ & $xz$ & $y \oplus (xz)$ & $x \leftrightarrow (y \oplus (xz))$ & $f(x,y,z)$ \\
\hline
0 & 0 & 0 & 1 & 0 & 0 & 1 & 1 \\
0 & 0 & 1 & 1 & 0 & 0 & 1 & 1 \\
0 & 1 & 0 & 1 & 0 & 1 & 0 & 0 \\
0 & 1 & 1 & 1 & 0 & 1 & 0 & 0 \\
1 & 0 & 0 & 0 & 0 & 0 & 0 & 1 \\
1 & 0 & 1 & 0 & 1 & 1 & 1 & 1 \\
1 & 1 & 0 & 0 & 0 & 1 & 1 & 1 \\
1 & 1 & 1 & 0 & 1 & 0 & 0 & 1 \\
\hline
\end{tabular}
\end{center}
\subsection{Analysis}
\label{sec:orgc147255}
The function evaluates to 0 when \(x=0\) and \(y=1\) (rows 3 and 4), and evaluates to 1 otherwise. Therefore:
\[
\boxed{f(x, y, z) \not\equiv 1 \text{ (not a tautology)}}
\]

The function can be expressed in DNF (Disjunctive Normal Form) by taking the minterms where \(f=1\):
\[
f(x,y,z) = \bar{x}\bar{y}\bar{z} \lor \bar{x}\bar{y}z \lor x\bar{y}\bar{z} \lor x\bar{y}z \lor xy\bar{z} \lor xyz
\]

\newpage
\section{Using K-maps to Simplify Boolean Expression}
\label{sec:orgc6c74e6}
Using K-maps simplify the following boolean expression:
\[
f(w, x, y, z)= \bar{w}xy\bar{z}+w\bar{x}yz +wx\bar{y}\bar{z}+\bar{w}xy\bar{z}+\bar{w}\bar{x}y\bar{z}+\bar{w}x\bar{y}z+w\bar{x}\bar{y}\bar{z}+\bar{w}xyz+w\bar{x}yz+wx\bar{y}z+\bar{w}xy\bar{z}+\bar{w}x\bar{y}\bar{z}
\]
\subsection{Identifying the Minterms}
\label{sec:org69322df}
First, we identify which minterms are present (removing duplicates):
\begin{itemize}
\item \(\bar{w}xy\bar{z}\) → 0110 → minterm 6
\item \(w\bar{x}yz\) → 1011 → minterm 11
\item \(wx\bar{y}\bar{z}\) → 1100 → minterm 12
\item \(\bar{w}\bar{x}y\bar{z}\) → 0010 → minterm 2
\item \(\bar{w}x\bar{y}z\) → 0101 → minterm 5
\item \(w\bar{x}\bar{y}\bar{z}\) → 1000 → minterm 8
\item \(\bar{w}xyz\) → 0111 → minterm 7
\item \(wx\bar{y}z\) → 1101 → minterm 13
\item \(\bar{w}x\bar{y}\bar{z}\) → 0100 → minterm 4
\end{itemize}

Minterms: 2, 4, 5, 6, 7, 8, 11, 12, 13
\subsection{Constructing the K-map}
\label{sec:orgec6fd93}
We fill in a 4-variable Karnaugh map:

\begin{center}
\begin{tabular}{|c|c|c|c|c|}
\hline
$wx \backslash yz$ & 00 & 01 & 11 & 10 \\
\hline
00 & 0 & 0 & 0 & 1 \\
01 & 1 & 1 & 1 & 1 \\
11 & 1 & 1 & 0 & 0 \\
10 & 1 & 0 & 1 & 0 \\
\hline
\end{tabular}
\end{center}

Where rows represent \(wx\): 00, 01, 11, 10 and columns represent \(yz\): 00, 01, 11, 10.

\newpage
\subsection{Grouping the Ones}
\label{sec:orga1c193a}
We identify the prime implicants by grouping adjacent 1s in the K-map. The goal is to form the largest possible groups (powers of 2) to minimize the final expression.

\textbf{Group 1} (size 4): Covers minterms 4, 5, 6, 7
\begin{itemize}
\item Row 01, all columns
\item Variables that remain constant: \(w=0, x=1\)
\item Prime implicant: \(\bar{w}x\)
\end{itemize}

\textbf{Group 2} (size 2): Covers minterms 12, 13
\begin{itemize}
\item Row 11, columns 00 and 01
\item Variables that remain constant: \(w=1, x=1, y=0\)
\item Prime implicant: \(wx\bar{y}\)
\end{itemize}

\textbf{Group 3} (size 2): Covers minterms 2, 6
\begin{itemize}
\item Column 10, rows 00 and 01
\item Variables that remain constant: \(w=0, y=1, z=0\)
\item Prime implicant: \(\bar{w}y\bar{z}\)
\end{itemize}

\textbf{Group 4} (size 2): Covers minterms 8, 12
\begin{itemize}
\item Column 00, rows 10 and 11
\item Variables that remain constant: \(w=1, y=0, z=0\)
\item Prime implicant: \(w\bar{y}\bar{z}\)
\end{itemize}

\textbf{Group 5} (size 1): Covers minterm 11
\begin{itemize}
\item This minterm cannot be grouped with others
\item Prime implicant: \(w\bar{x}yz\)
\end{itemize}
\subsection{Simplified Expression}
\label{sec:org7617c25}
Combining all prime implicants that are necessary to cover all minterms:
\[
\boxed{f(w,x,y,z) = \bar{w}x + wx\bar{y} + \bar{w}y\bar{z} + w\bar{y}\bar{z} + w\bar{x}yz}
\]

Note: \(\bar{w}y\bar{z}\) is essential for minterm 2, \(w\bar{y}\bar{z}\) is essential for minterm 8, and \(w\bar{x}yz\) is essential for minterm 11.
\end{document}
